\input{../header}
\usepackage{parskip}
\everymath{\displaystyle}
\rhead{Your name: \rule{8cm}{0.15mm}}

\begin{document}
%


%\onehalfspacing
\allowdisplaybreaks
%##################################################################
\section{Chapter 1 checkpoint!}

Hello and welcome to your first checkpoint! Here come six questions, one about each of the learning targets from Chapter 1. This is your scorecard:

\begin{center}
    \begin{tabular}{|m{3.75cm}|*{6}{m{1.75cm}|}} \hline
        Learning target: & P1 & P2 & L1 & L2 & L3 & L4 \\\hline
        Your confidence level before starting (0-5): & &&&&&\\\hline
        Your confidence level after the quiz (0-5): & &&&&&\\\hline
        The mark you earned on this attempt: 
        & Success! \newline Revise! \newline Try again!
        & Success! \newline Revise! \newline Try again!
        & Success! \newline Revise! \newline Try again!
        & Success! \newline Revise! \newline Try again!
        & Success! \newline Revise! \newline Try again!
        & Success! \newline Revise! \newline Try again! \\\hline

    \end{tabular}
\end{center}

Before anything else, please do the following:
\begin{itemize}
    \item Rank your confidence from 0-5 on each of the learning targets. 5 means ``I could teach a whole class about this;'' 0 means ``I am genuinely not sure I have heard these words before.''
    \item Rip all the pages apart.
    \item Write your name on this page and on each of the other pages of the quiz.
\end{itemize}

Then do the quiz! Some reminders:
\begin{itemize}
    \item Open notes, closed computer.
    \item If you need more room to write, use the back of the same learning target page, or ask me for some scratch paper.
    \item Read the questions carefully and make sure you're answering each part.
    \item Show all your work and explain all your thinking!
\end{itemize}

When you are done:
\begin{itemize}
    \item Rank your confidence from 0-5 on each of the learning targets. 5 means ``I absolutely nailed that question for sure;'' 0 means ``oof, I definitely didn't get that one.''
    \item Make double sure your name is on every page, including any scratch paper.
    \item Hand in your work, separated by learning target.
\end{itemize}

Have fun and do your best! I believe in u $\heartsuit$


%%%%%%%%%%%%%%%%%%%%%%%%%%%%%%%%%%%%%%%%%%%%%%%%%%%%%%%%%
\pagebreak
%%%%%%%%%%%%%%%%%%%%%%%%%%%%%%%%%%%%%%%%%%%%%%%%%%%%%%%%%
\section{Learning target P1, version 1}
Consider the following proposition:

For all natural numbers $n$, if $n^{2}$ is even, then $n$ is even. 

Is the following claimed proof valid? Use Toulmin analysis to decide.

\hrulefill

    \begin{proof} Consider any $n \in \mathbb{N}$. Assume that $n^{2}$ is even. 
    
    By definition, $n^{2} = 2k$ for some $k \in \mathbb{N}$.
    
    Let's take square roots of each side giving us $\sqrt{n^{2}}=\sqrt{2k}$. 
    
    Simplifying, $n=\sqrt{2k}$.  
    
    Multiplying by $2$ top and bottom inside the square root, $n=\sqrt{\frac{2\cdot 2k}{2}}$. 
    
    So, $n=\sqrt{\frac{4\cdot k}{2}}$.  
    
    Pulling $4$ out of the square root, $n=2\cdot\sqrt{\frac{k}{2}}$.
    
    Let $m=\sqrt{\frac{k}{2}}$.
    
    Thus $n=2m$.
    
    Therefore, $n$ is an even number.
    
    So, our theorem is proved.
    \end{proof}
%%%%%%%%%%%%%%%%%%%%%%%%%%%%%%%%%%%%%%%%%%%%%%%%%%%%%%%%%
\pagebreak
%%%%%%%%%%%%%%%%%%%%%%%%%%%%%%%%%%%%%%%%%%%%%%%%%%%%%%%%%
\section{Learning target P2, version 1}

The following definitions are extremely made up and don't mean anything (but they look fun!).

\textbf{Definition}: A function is \textit{viscous} if its geodesic curvature $\gamma(f)$ is nonzero.

\textbf{Definition}: A function is \textit{laminar} if its intrinsic cofunction $\tilde{f}$ is differentiable.

Consider the following proposition: Every laminar function is viscous.

\begin{enumerate}[leftmargin=0pt]
    \item Write a proof framework for a \textit{direct} proof of this proposition.
    \vfill
    \item Write a proof framework for a \textit{proof by contrapositive.}
    \vfill
    \item Write a proof framework for a \textit{proof by contradiction.}
    \vfill
    \item (Bonus!) If for some reason you did not believe this proposition was true (and thus that none of these proof frameworks is appropriate), how would you show it?
\end{enumerate}
%%%%%%%%%%%%%%%%%%%%%%%%%%%%%%%%%%%%%%%%%%%%%%%%%%%%%%%%%
\pagebreak
%%%%%%%%%%%%%%%%%%%%%%%%%%%%%%%%%%%%%%%%%%%%%%%%%%%%%%%%%
\section{Learning target L1, version 1}

Consider these two very similar-looking statements: 

$\exists x\in \mathbb{Z}\ \forall y \in \mathbb{Z}, x + y = 0$

$\forall y \in \mathbb{Z}\ \exists x\in \mathbb{Z}, x + y = 0$

\begin{enumerate}
    \item Translate both statements into human words.
    \vfill
    \item Precisely one of these two statements is true. Which is it, and why?\\ (Moral: the order of quantifiers is super important.)
    \vfill
    \item (Bonus!) If $\mathbb{Z}$ is replaced by $\mathbb{N}$, is this statement still true?
\end{enumerate}

\iffalse
%%%%%%%%%%%%%%%%%%%%%%%%%%%%%%%%%%%%%%%%%%%%%%%%%%%%%%%%%
\pagebreak
%%%%%%%%%%%%%%%%%%%%%%%%%%%%%%%%%%%%%%%%%%%%%%%%%%%%%%%%%
\section{Learning target L1, version 0}

Okay, so here's the real definition of a function: A \textbf{function} $f:X\to Y$ is a rule that assigns each input exactly one output. \textbf{Inputs} are elements of the \textbf{domain} $X$, and \textbf{outputs} are elements of the \textbf{codomain} $Y$.

\begin{enumerate}
    \item Write the definition of a function in logical symbols with proper quantifiers.
    \vfill
    \item An \textbf{injective} function (also called \textbf{one-to-one}) assigns each output to exactly one input. Translate that to properly quantified logical symbols.
    \vfill
    \item A \textbf{surjective} function (also called \textbf{onto}) assigns an input to every possible output. Logical symbols for this?
    \vfill
\end{enumerate}
\fi

%%%%%%%%%%%%%%%%%%%%%%%%%%%%%%%%%%%%%%%%%%%%%%%%%%%%%%%%%
\pagebreak
%%%%%%%%%%%%%%%%%%%%%%%%%%%%%%%%%%%%%%%%%%%%%%%%%%%%%%%%%
\section{Learning target L2, version 1}

Write each of the following statements in ``if\ldots, then \ldots'' form. (Note that some of them are true and some of them are not.)

\begin{enumerate}
    \item Every continuous function is differentiable. \vfill
    \item A shape is a square only if it is a rectangle. \vfill
    \item Cooking is necessary for eating salmon. \vfill
    \item All dogs go to heaven. \vfill
    \item (The converse of ``all dogs go to heaven''.) \vfill
    \item (The inverse of ``all dogs go to heaven''.) \vfill
    \item (The contrapositive of ``all dogs go to heaven''.) \vfill
    
\end{enumerate}

%%%%%%%%%%%%%%%%%%%%%%%%%%%%%%%%%%%%%%%%%%%%%%%%%%%%%%%%%
\pagebreak
%%%%%%%%%%%%%%%%%%%%%%%%%%%%%%%%%%%%%%%%%%%%%%%%%%%%%%%%%
\section{Learning target L3, version 1}

Simplify each of the following:

\begin{enumerate}
    \item $\lnot(P\to Q)$ \vfill
    \item The negation of the \textbf{converse} of $P\to Q$\vfill
    \item $\lnot((P\to Q) \lor (Q\to P))$\vfill
    \item $\neg \exists x \forall y (\neg O(x) \vee E(y))\text{.}$\vfill
\end{enumerate}

%%%%%%%%%%%%%%%%%%%%%%%%%%%%%%%%%%%%%%%%%%%%%%%%%%%%%%%%%
\pagebreak
%%%%%%%%%%%%%%%%%%%%%%%%%%%%%%%%%%%%%%%%%%%%%%%%%%%%%%%%%
\section{Learning target L4, version 1}

I wonder how the implication sign distributes over the logical connectives $\lor$ and $\land$.

\begin{enumerate}
    \item Are the statements $P \to (Q\lor R)$ and $(P\to Q) \lor (P \to R)$ logically equivalent? \\
    Use a truth table to decide.
    \vfill
    \item What about the statements $P \to (Q\land R)$ and $(P\to Q) \land (P \to R)$? Make another truth table.
    \vfill
\end{enumerate}
\end{document}